\chapter{Back End}
Code Generation consists of Instruction Selection, Instruction scheduling and Register allocation. All of them are NP-complete for exact graph-based versions.
\section{Instruction Selection}
\begin{itemize}
	\item The process of mapping IR code to target machine instructions
	\item pattern matching problem: Tree-pattern matching or Peephole optimization
	\item costs depend highly on ISA provided instructions (RISC, CISC, Stack machines)
	\item compiler writer should associate costs to instruction and built up a truly reflecting cost model for efficient code
\end{itemize}

\subsection{Tree Pattern-Matching}
\begin{enumerate}
	\item Express target instruction set as a set of trees
	\item Build AST from source
	\item Find the minimal cost tiling, that implements every operation and all tiles connect to their neighbors (i.e. the AST node is covered by the current and a neighbour tile)
\end{enumerate}
\img{2/isel_0}
\img{2/isel_1}
\img{2/isel_2}
\img{2/isel_3}
\img{2/isel_4}
\img{2/isel_5}
\img{2/isel_6}

\subsection{Peephole Optimization}
\img{2/peephole_0}
\img{2/peephole_1}
\img{2/peephole_2}
\img{2/peephole_3}
\img{2/peephole_4}
\img{2/peephole_5}

\section{Instruction Scheduling}
\img{2/isched_0}
\img{2/isched_1}
\textbf{An Operation} is a single opcode with operands. \textbf{An Instruction} is an aggregation of one or more operations that all issue in the same cycle. \\
Instruction scheduling is the process of reordering the code to decrease the running time. It must preserve the meaning of the code , it shall minimize the execution time by avoiding stalls or nops and it should avoid increasing value lifetimes past the point where additional register spills are necessary.
\img{2/isched_2}
\img{2/isched_3}
\img{2/isched_4}

\subsection{Local List Scheduling}
\img{2/local_list_0}
\img{2/local_list_1}
\img{2/local_list_2}
\img{2/local_list_3}
\img{2/local_list_4}
\img{2/local_list_5}
\img{2/local_list_6}
\img{2/local_list_7}
\img{2/local_list_8}


\subsection{Regional List Scheduling}
\img{2/regional_list_0}
\img{2/regional_list_1}
\img{2/regional_list_2}
\img{2/regional_list_3}
\img{2/regional_list_4}

\section{Register Allocation}
\img{2/register_0}
\img{2/register_1}
\img{2/register_2}
\img{2/global_0}
\img{2/global_1}
\img{2/global_2}


\subsection{Local Register Allocation}
\paragraph{Top-Down}
\img{2/register_3}
\img{2/register_4}
\img{2/register_5}
\paragraph{Bottom-Up}
\img{2/register_6}
\img{2/register_7}
\img{2/register_8}
\img{2/register_9}
\img{2/register_10}
\img{2/register_11}

\subsection{Global Register Allocation}
\img{2/global_3}
\img{2/global_4}
\img{2/global_5}
\img{2/global_6}
\img{2/global_7}
\img{2/global_8}

\paragraph{Top-Down}
\img{2/global_9}
\img{2/global_10}
\img{2/global_11}
\img{2/global_12}

\paragraph{Bottom-Up}
\img{2/global_13}
\img{2/global_14}
\img{2/global_15}
\img{2/global_16}
\img{2/global_17}
\img{2/global_18}